% Introduction
\documentclass[../review.tex]{subfiles}

\begin{document}
% give context
% present issue/topic to be discussed
% give an overview of how the review will be organised
\section{Introduction}
Regarding performance, microkernel systems have had a bad reputation in comparison to monolithic kernel systems. This idea seems to be firmly supplanted within the psyche of the operating systems space, especially since the MACH 3.0 debacle (Bershad \& Chen, 1993).

Microkernel Goes General: Performance and Compatibility in the HongMeng Production Microkernel (Chen, 2024) introduces a production "microkernel" system that is able provide sufficient performance for real world general operating system usage. However, this paper outlines a series of optimisations and design choices which makes a departure from a strictly purist concept of a microkernel, which had been argued to be essential for good performance by Jochen Liedtke (On $\mu$-kernel construction). Furthermore, The LionsOS preprint suggests that this departure may not be required to emulate or even beat the performance of Linux.

This literature review will revolve around the design choices made by Chen et al and whether or not it was necessary to make these choices and whether these choices were well informed choices, keeping in mind previous and upcoming research. I will begin by analysing their implementation of isolation zones as a way to bypass IPC and also their decision to use 

\newpage
\end{document}